\documentclass[10pt,oneside]{article}
\usepackage{geometry}
\usepackage[T1]{fontenc}

\pagestyle{empty}
\geometry{letterpaper,tmargin=1in,bmargin=1in,lmargin=0.9in,rmargin=0.9in,headheight=0in,headsep=0in,footskip=.3in}

\setlength{\parindent}{0in}
\setlength{\parskip}{0in}
\setlength{\itemsep}{0in}
\setlength{\topsep}{0in}
\setlength{\tabcolsep}{0in}

% Name and contact information
\newcommand{\name}{Yang Liu}
\newcommand{\addr}{}
\newcommand{\phone}{(+86) 13162516663}
\newcommand{\email}{yang.liu5@ucalgary.ca}


% This defines how the name looks
\newcommand{\bigname}[1]{
	\begin{center}\fontfamily{phv}\selectfont\Huge\scshape#1\end{center}
}

% A ressection is a main section (<H1>Section</H1>)
\newenvironment{ressection}[1]{
	\vspace{4pt}
	{\fontfamily{phv}\selectfont\Large#1}
	\begin{itemize}
	\vspace{3pt}
}{
	\end{itemize}
}

% A resitem is a simple list element in a ressection (first level)
\newcommand{\resitem}[1]{
	\vspace{-4pt}
	\item \begin{flushleft} #1 \end{flushleft}
%	\item  #1 
}

% A ressubitem is a simple list element in anything but a ressection (second level)
\newcommand{\ressubitem}[1]{
	\vspace{-1pt}
	\item \begin{flushleft} #1 \end{flushleft}
%	\item  #1 
}

% A resbigitem is a complex list element for stuff like jobs and education:
%  Arg 1: Name of company or university
%  Arg 2: Location
%  Arg 3: Title and/or date range
\newcommand{\resbigitem}[3]{
	\vspace{-5pt}
	\item
	\textbf{#1}---#2 \
	\textit{#3}
}

% This is a list that comes with a resbigitem
\newenvironment{ressubsec}[3]{
	\resbigitem{#1}{#2}{#3}
	\vspace{-2pt}
	\begin{itemize}
}{
	\end{itemize}
}

% This is a simple sublist
\newenvironment{reslist}[1]{
	\resitem{\textbf{#1}}
	\vspace{-5pt}
	\begin{itemize}
}{
	\end{itemize}
}



%%%%%%%%%%%%%%%%%%%%%%%%%%%%%%%%%%%%%%%%%%%%%%%%%%%%%%%%%
% Now for the actual document:

\begin{document}

\fontfamily{ppl} \selectfont

% Name with horizontal rule
\bigname{Yang Liu}

\vspace{-8pt} \rule{\textwidth}{1pt}

\vspace{-1pt} {ICT 719, University of Calgary, 2500 University Dr NW. \hfill yang.jace.liu@linux.com}

\vspace{-1pt} {Calgary, Alberta, Canada}

\vspace{-1pt} {(+1) (403)667-6663}

\vspace{8 pt}

\newcommand{\textsharp}{$\sharp$}

%%%%%%%%%%%%%%%%%%%%%%%%
\begin{ressection}{Education}
	\begin{ressubsec}{University of Calgary}{Calgary, AB, Canada \hfill \textbf{Jan. 2019 - Present}}{}
		\ressubitem{M.Sc in Computer Science}
	\end{ressubsec}
	\begin{ressubsec}{Tongji University}{Shanghai, China \hfill \textbf{Sept. 2013-Jun. 2017}}{}
		\ressubitem{B.S. in Information Security, Computer Science and Technology Institution. GPA: 88.65/100. Rank: 4th of 52. }
	\end{ressubsec}
\end{ressection}


\begin{ressection}{Internship and Work}
	\begin{reslist}{SNLab \hfill Oct. 2016 - Aug. 2018}
		\ressubitem{Developed an Intellij IDEA plugin for Yang modeling language, supporting syntax analysis and highlights.}
        \ressubitem{Developed a complete IDE to support the whole SDN APP development on Maple(SIGCOMM 2014) and FAST(SIGCOMM Poster).}
		\ressubitem{Developed Unicorn together, a infrastructure to minimize time consumption of transmissions of big files, whose paper published in SuperComputing 2017 INDIS workshop and demo presented in SuperComputing 2017 conference}
	\end{reslist}
\end{ressection}

\begin{ressection}{Publications}
	\ressubitem{Unicorn: Unified resource orchestration for multi-domain, geo-distributed data analytics; Future Generation Computer Systems, Vol 93, 2019, Pages 188-197}
	\ressubitem{Fine-grained, multi-domain network resource abstraction as a fundamental primitive to enable high-performance, collaborative data sciences; Supercomputing 2018. 2018}
	\ressubitem{Fine-Grained, Multi-Domain Network Resource Abstraction as a Fundamental Primitive to Enable High-Performance, Collaborative Data Sciences; ACM SIG’COMM 2018 Posters, Demos, and Student Research Competition. 2018}
	\ressubitem{Unicorn: Unified Resource Orchestration for Multi-Domain, Geo-Distributed Data Analytics; Supercomputing 2017 INDIS workshop. 2017}
\end{ressection}
	
%%%%%%%%%%%%%%%%%%%%%%%%
\begin{ressection}{Project Experience}
%	\begin{reslist}{Tic-tac-toe Artificial Intelligence \hfill July. 2015 - Oct. 2015}
%		\ressubitem{Developed Graphical User Interface (front end) by Qt Quick 5, and back end by Python3.}
%		\ressubitem{Connected front end and back end in C++ and controlled the logic of front end in Javascript.}
%		\ressubitem{Applied game tree the main algorithm and used a valuation function to calculate the priority after comparing the result of each step.}
%		\ressubitem{Optimized the program performance with $\alpha-\beta$ pruning algorithm.}
%	\end{reslist}
%	\begin{reslist}{MIPS Microcomputer \hfill Oct. 2015 - Dec. 2015}
%		\ressubitem{Implemented a microcomputer based on MIPS Instruction set, developed in Verilog and tested on FPGA (Xilinx Nexys 3). }
%		\ressubitem{Equipped with a 5-stage-pipeline CPU in Harvard Architecture to reach a highest clock frequency.}
%		\ressubitem{Developed a Tic-Tac-Toe game written in MIPS assembly language running on the microcomputer.}
%	\end{reslist}
	\begin{reslist}{Indoor Autonomous Patrol Robot \hfill Apr. 2016 - July. 2016}
		\ressubitem{Based the control program on OpenCV and ROS, running on Linux and communicating in socket.}
		\ressubitem{Equipped with autonomous patrol and simultaneous suspecious items detection.}
		\ressubitem{Enabled users to control the robot to inspect surroundings, manually by a specific joystick.}
	\end{reslist}
	% \begin{reslist}{SDN Visual Programming IDE \hfill Mar. 2017 - July. 2017}
	% 	\ressubitem{Based on a browser-based IDE, running in Docker, accessed from anywhere in any browser.}
	% 	\ressubitem{With design of the first SDN visual programming model in the world.}
	% 	\ressubitem{With design of the first extensible, strong typed visual programming langauge.}
	% 	\ressubitem{Deployed in Caltech LHC/CMS testbed.}
	% \end{reslist}
	\begin{reslist}{Unicorn \hfill Aug. 2017 - Nov. 2018}
		\ressubitem{Designed for unified resource scheduling for multi-domain, geo-distributed networks.}
		\ressubitem{Received by Supercomputing 2018 Proceedings.}
		\ressubitem{Presented in Supercomputing 2017 Conference and Supercomputing 2018 Conference in demonstration.}
	\end{reslist}
\end{ressection}

%%%%%%%%%%%%%%%%%%%%%%%%

% \begin{ressection}{Skills}
% 	\resitem{Programming Languages: C, C++, C$^\sharp$, Java, Python, JavaScript, SQL}
% 	\resitem{Working knowledge in Linux, Visual Studio, Vim, Intellij IDEA, Git}
% 	\resitem{Familiar with Developments in Single-chip Microcomputers, Embedded Platforms and Software Defined Networking(SDN)}
% \end{ressection}

%%%%%%%%%%%%%%%%%%%%%%%%



\begin{ressection}{Awards and Honors}
	\resitem{1st Prize in Intel Cup Undergraduate Electronic Design Contest - Embedded System Design Invitational Contest (top 13 in 160 teams all over the world) \hfill July. 2016}
	\resitem{3rd Prize in Shanghai in National Undergraduate Electronic Design Contest \hfill {June. 2015}}
\end{ressection}

\end{document}
